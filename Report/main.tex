\documentclass[12pt]{report}

% ----- PREAMBLE -----
\usepackage{amsmath}
\usepackage{amssymb}
\usepackage{graphicx}
\usepackage{geometry}
\usepackage{setspace}
\usepackage{hyperref}

\title{\textbf{Soft Robotic Gait Optimization Using Deep Learning}}
\author{Joel Valley}
\date{\today}

% ----- DOCUMENT -----
\begin{document}

\maketitle
\tableofcontents
\newpage

\chapter{Outline}
\section{Background}
The development of an untethered soft robot that uses 
electromagnetic leg actuation to drive locamotive action 
requires specific input parameters to create a desired
gait. The desired optimized variable for this project is the
robot's forward velocity. The goal is to not only predict the 
maximum velocity that the robot can move, but also the input 
parameters controlling the gait to achieve this desired maximum. 

\begin{figure}[h]
\centering
\includegraphics[width=0.5\textwidth]{robot.jpg}
\caption{Soft Robot Prototype}
\label{fig1}
\end{figure}

As shown in Figure \ref{fig1}, the soft robot consists of
a rigid plastic shell with to electromagnetic solenoids connected
on either side. The soft portion of the robot is its silicone
legs which are embedded with permenant magents. At the ends of
the silicone legs are plastic feet with silicone embedded on only
half of each foot, which allows for directional friction to ensure
the robot is propelled forward. By pulsing the electromagnetic
solenoids on each leg, the robot builds oscillatory momentum,
and this rocking motion is what propels the prototype forward.

\begin{figure}[h]
\centering
\includegraphics[width=0.8\textwidth]{Leg_Pulse_Example_Plot.png}
\caption{Leg Pulse Waveform Example Plot}
\label{fig2}
\end{figure}

The current hypothesis is that three main parameters contribute
the most to the output velocity: first leg pulse width, second
leg pulse width, and the offset between the waveforms. As seen
in Figure \ref{fig2}, the parameters from here onward denoted as
x, y and z will be the two leg pulse durations, and the pulse 
offset, respectively. Each leg has its own square waveform where
the current passes through the solenoid one way, then after the
set pulse duration, the motor drivers will be used to change
the direction of the current, thus switching the electromagnet's
polarity. The offset, z, is the time at which one leg's pulse
waveform lags the other. By varying these parameters, the forward
velocity of the robot should change, and there should be some
value for each of these parameters that will maximize that
velocity. It is important to note that Figure \ref{fig2} is only
for illustrative purposes.



\chapter{Offline Synthetic Data Optimization}
\section{Introduction}

To familiarize ourselves with the concepts of what is to be 
implemented on the prototype, the first phase of the gait
optimization is performing the optimization offline using 
synthetic data. A 3-dimensional function is used to generate a 
dataset of speed values $(w)$ using random inputs $(x, y, z)$. 
This dataset is then passed to a multiple regression model that
is to learn how to map the inputs to the outputs. This can then
be taken a step further by optimizing an initial guess of the 
input parameters and performing gradient ascent on the 
model's predictions to find the maximum value of the model's 
predictions to locate the maximum predicted output. Once the 
model's predictions are maximized, the remaining optimized input
represents the model's estimate of the parameters that produce 
the maximum output. This estimated maximum can then be compared 
to the true maximum of the underlying function (if known), 
allowing the error and accuracy of the model to be evaluated.

\pagebreak
\section{Paraboloid Dataset}
\subsection{Dataset Generation}
The first synthetic dataset used to perform multiple regression 
is a simple concave-down 4D paraboloid with some shifting to have 
the vertex at a known, non-zero point.
\begin{equation}
w = -((x-200)^2+(y-150)^2+(z-25)^2)+20+\epsilon
\label{w}
\end{equation}
where:
\begin{description}
\item\hspace{2em}{$w$ = speed of the robot}
\item\hspace{2em}{$x$ = first leg pulse width}
\item\hspace{2em}{$y$ = second leg pulse width}
\item\hspace{2em}{$z$ = leg pulses waveforms offset}
\item\hspace{2em}{$\epsilon$ = random noise $[-0.1, 0.1]$}
\end{description} 
\vspace{1em}

From the mathematical model in \eqref{w}, it is evident that
the maximum value of the paraboloid is $w\in [20-|\epsilon_{max}|, 20+|\epsilon_{max}|]$,
and occurs at the vertex $(200, 150, 25)$.

Five-hundred sample points were generated using a normal 
distribution with a mean at the vertex of the paraboloid, and
a standard deviation of 70 for each input. These values were chosen
so that more datapoints near the maximum of the paraboloid would
be generated, allowing the model to more accurately predict the 
maximum. A medium value for the standard deviation was chosen so that
the model will still learn a general pattern for the data and not
solely values near the peak.

Before passing the dataset to the neural network, it was normalized
centered at zero using scikit-learn's StandardScaler. The data was 
split into a 80:20 train/test split, and the data was batchified 
for improved model training.

\subsection{Model Architecture}
The architecture of the neural network used to learn this
dataset is a feed-forward MLP neural network designed for 
multiple regression, with four linear layers and three GELU layers. 
The GELU activation function was chosen through experimentation, 
and was used instead of ReLU to avoid dying ReLU, since our data 
is normalized and centered at zero, meaning we can have negative 
values. The model was trained using three input parameters 
$(x, y, z)$, 1024 hidden units, and one output $(w)$.

\subsection{Model Training}
The model was trained for 2500 epochs using an ADAM optimizer for
faster convergence, and MSE loss was chosen for its sensitivity to
outliers. The learning rate of the optimizer is $1\times10^{-5}$,
and was selected through testing trial-and-error. The model
architecture and training parameters will be optimized in future
sections using a ML pipeline library of sorts.

\subsection{Results}
After 2500 epochs of training, the model has a training loss of
$0.0000018$, and a test loss of $0.0003811$. The loss decreased
very fast in training, however, since the data is normalized, it
was noticed that even a tiny amount of error in the scaled values
results in a large error when the predictions are upscaled back to
their original sizes. This is why the model was trained for so many
epochs on a relatively simple dataset. 

\begin{figure}[h]
\centering
\includegraphics[width=0.8\textwidth]{Loss_Curve.png}
\caption{Loss Curves}
\label{fig3}
\end{figure}

The model was then used to predict some points on the paraboloid
and compare them to the actual values.

\begin{table}[h]
\centering
\begin{tabular}{|c|c|c|c|c|c|c|}
\hline
$x$ & $y$ & $z$ & Pred (Scaled) & Actual (Scaled) & Pred (Real) & Actual (Real) \\ \hline
200 & 150 & 25 & 1.19187 & 1.19070 & 31.95191 & 20.07000 \\ \hline
0 & 0 & 0 & -4.86145 & -5.03438 & -61351.35938 & -63104.92000 \\ \hline
400 & 300 & 50 & -4.93359 & -5.03438 & -62082.93359 & -63104.91000 \\ \hline
150 & 78 & 100 & -0.12526 & -0.12177 & -13324.37207 & -13288.96000 \\ \hline
\end{tabular}
\caption{Sample Model Predictions}
\label{table1}
\end{table}

As shown in Table \ref{table1}, the scaled real values are quite
close to the actual values calculated using Equation \ref{w},
however, there is still some unecessary error. Methods to further
improve upon this error will be implemented and experimented with
in future sections. For this initial test, these values are adequate.

After the model learned the dataset, to test if the model could
be used to accurately predict the input values that create the
maximum speed, a tensor of random input values $(x, y, z)$ is
optimized using another ADAM optimizer, and the loss function
is the negative of the model's predictions to perform gradient
ascent. After the random input tensor is optimized, the resulting
input parameters to create the optimal speed predicted by the model
were found to be:

\[
(x, y, z)
\]

\section{Spherical Dataset}
\subsection{Dataset Generation}

\end{document}
